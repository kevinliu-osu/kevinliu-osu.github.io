\documentclass[overlapped,line,letterpaper]{res1}

\usepackage{ifpdf}
\usepackage{fancyhdr}
\usepackage{amsmath, amsthm, amssymb}
\usepackage{url}
\usepackage{enumitem}

\ifpdf
  \usepackage[pdftex]{hyperref}
\else
  \usepackage[hypertex]{hyperref}
\fi

\hypersetup{
  letterpaper,
  colorlinks,
  urlcolor=black,
  pdfpagemode=none,
  pdftitle={Curriculum Vitae},
  pdfauthor={Jia Liu},
  pdfcreator={$ $Id: cv-us.tex,v 1.23 2006/06/21 00:14:55 jrblevin Exp $ $},
  pdfsubject={Curriculum Vitae},
  pdfkeywords={economics microeconomics econometrics empirical urban
    game theory applied mathematics differential equations TANH method
    hyperbolic tangent centroid decomposition updating singular value
    decomposition linear algebra microeconomics linux unix german}
}

\def\Cplusplus{{\rm C\raise.5ex\hbox{\small ++}}}

\addtolength{\textheight}{-0.8cm} %\addtolength{\voffset}{-0.2in}

\newenvironment{list1}{
  \begin{list}{\ding{113}}{%
      \setlength{\itemsep}{0in}
      \setlength{\parsep}{0in} \setlength{\parskip}{0in}
      \setlength{\topsep}{0in} \setlength{\partopsep}{0in}
      \setlength{\leftmargin}{0.17in}}}{\end{list}}\newenvironment{list2}{
  \begin{list}{$\bullet$}{%
      \setlength{\itemsep}{0in}
      \setlength{\parsep}{0in} \setlength{\parskip}{0in}
      \setlength{\topsep}{0in} \setlength{\partopsep}{0in}
      \setlength{\leftmargin}{0.2in}}}{\end{list}}

%%===========================================================================%%

\setlist[enumerate,1]{leftmargin=\dimexpr 26pt-.2in}

\begin{document}

%---------------------------------------------------------------------------
% Document Specific Customizations

% Make lists without bullets and with no indentation
%\setlength{\leftmargini}{0em}
%\renewcommand{\labelitemi}{}

%% Use large bold font for printed name at top of pages
%\renewcommand{\namefont}{\large\textbf}
%
%\renewcommand{\headrulewidth}{0pt}

% Use large bold font for printed name at top of pages
\renewcommand{\namefont}{\large\textbf}

\pagestyle{fancy} \headheight0pt \lfoot{\hspace{-.5in}{{\bf Search \#: 85352}}}
 \rfoot{\bf \thepage} \cfoot{}

\renewcommand{\headrulewidth}{0pt}


%---------------------------------------------------------------------------
\name{\LARGE Jia (Kevin) Liu}

\begin{resume}

%\section{\bf \large Contact Information}
%\vspace{.1in}
%\begin{ncolumn}{2}
%    608 Dreese Labs       & \hspace{-.5in} E-mail: {\tt liu.1736@osu.edu}, \,\, Phone: (540) 239-6008 \\
%  Columbus, OH 43210                      & \hspace{-.5in} Web: {\tt \verb+http://www2.ece.ohio-state.edu/~liu/+} \\
%\end{ncolumn}
%
%%---------------------------------------------------------------------------
%
%\section{\bf \large Research Interests}
%\vspace{.28in}
%\begin{list2}
%\item Communications, Networking, Signal Processing, Control
%\item Distributed Control and Optimization, Data Analytics
%\item Mathematical Optimization Theory, Algorithm Design
%\end{list2}
%
%
%
%\section{\bf \large Education}
%\vspace{.1in}
%%{\bf Ph.D. in Electrical and Computer Engineering}, Dec. 2009 (expected) \\ %Feb. 2010 (expected) \\
%{\bf Ph.D. in Electrical and Computer Engineering}, Feb. 2010 \\
%{\bf Virginia Tech}, Blacksburg, VA \\
%{\em Dissertation Topic:} ``MIMO-based Wireless Networks: Modeling and Optimization''
%%{\em Advisor:} Prof. Y. Thomas Hou
%%GPA: 3.97/4.00
%
%{\bf M.S. in Electrical Engineering}, March 1999 \\
%{\bf South China University of Technology}, Guangzhou, Guangdong, P. R. China
%%{\em Advisor:} Prof. Gang Wei
%%Summa Cum Laude, GPA: 3.9/4.0
%
%{\bf B.S. in Electrical Engineering}, July 1996 \\
%{\bf South China University of Technology}, Guangzhou, Guangdong, P. R. China
%%Cum Laude, GPA: 3.8/4.0
%
%{\bf B.S. in Computer Science}, July 1996 \\
%{\bf South China University of Technology}, Guangzhou, Guangdong, P. R. China
%%Cum Laude, GPA: 3.8/4.0
%
%
%%---------------------------------------------------------------------------
%\section{\bf \large Research and Working Experience}
%
%\vspace{.1in}
%{\bf Assistant Professor, Research}, {The Ohio State University, November 2014 - Present} \\
%\vspace{-.16in}\\
%{\em Research Areas:} Wireless communications, cloud computing, data analytics, smart grid \\
%\vspace{-.2in}
%
%{\bf Postdoctoral Researcher}, {Dept. of ECE, The Ohio State University, April 2013 - October 2014} \\
%\vspace{-.16in}\\
%{\em Research Areas:} Wireless communications, cloud computing, data analytics, smart grid\\ 
%\smallskip
%{\em Advisor:} Prof. Ness B. Shroff \\
%\vspace{-.2in}
%
%%{\bf Postdoctoral Researcher}, {Dept. of FABE, The Ohio State University, April 2010 - April 2013} \\
%%\vspace{-.16in}\\
%%{\em Research Areas:} Wireless sensor networks in swine barns, Indoor wireless sensor networks \\
%%\smallskip
%%{\em Advisors:} Prof. Lingying Zhao and Prof.~Scott Shearer \\
%%\vspace{-.2in}
%
%
%{\bf Graduate Research Assistant}, {Virginia Tech, September 2003 - February 2010} \\
%\vspace{-.16in}\\
%{\em Research Topic:} Cross-Layer Design for MIMO-Based Wireless Networks\\
%{\em Advisor:} Prof. Y. Thomas Hou \\
%\vspace{-.2in}
%
%{\bf Software Engineer Summer Intern}, {Mercury Computer Systems, Inc., Chelmsford, MA, June 2005 - August 2005} \\
%\smallskip
%{\em Project:} Developed Linux drivers for Rapid IO$^\circledR$ bus (used in Mercury's cognitive radio communications systems) \\
%\vspace{-.2in}
%
%
%{\bf Member of Technical Staff}, {Bell Labs, Lucent Technologies, May 2000 - January 2003} \\
%\medskip
%{\em Responsibility:} Development of cdma2000-1x/1xEV-DO/1xEV-DV standards in China \\
%\vspace{-.2in}
%
%{\bf System Engineer}, {Lucent Technologies, Guangzhou, P. R. China, March 1999 - May 2000} \\
%\vspace{-.16in} \\
%{\em Responsibility:} Technical support for Lucent's ATM/Frame Relay/Gigabit Ethernet switches \\
%%{\em Supervisor:} Xiaohui Shen \\
%\vspace{-.2in}
%
%
%%---------------------------------------------------------------------------
%%\vspace{.1in}
%\section{\bf \large Awards and Honors}
%\vspace{.2in}
%\begin{enumerate}
%%[leftmargin=-.01in]
%%\setlength{\leftmargin}{-100in}
%%\item {\bf AFRL 2015 Visiting Faculty Research Program (VFRP) Award}
%\item {\bf INFOCOM 2013 Best Paper Runner-up Award}: IEEE International Conference on Computer Communications (INFOCOM) 2013, (one best paper and two runner-ups were awarded out of 1600+ submissions, acceptance rate 17\%)
%
%\item {\bf IEEE ComSoc Technology News, 08/2013}, featuring my research on second-order distributed cross-layer optimization for wireless networks. URL: \url{http://www.comsoc.org/ctn/distributed-cross-layer-optimization-wireless-networks-second-order-approach.}
%
%\item {\bf INFOCOM 2011 Best Paper Runner-up Award}: IEEE International Conference on Computer Communications (INFOCOM) 2011, (one best paper and one runner-up were awarded out of 1800+ submissions, acceptance rate 15\%)
%
%\item {\bf ICC 2008 Best Paper Award}, IEEE International Conference on Communications (ICC) 2008
%
%\item {\bf China National Award for Outstanding Ph.D. Students Abroad 2008} (300 awardees across all disciplines worldwide)
%
%\item Co-recipient of Bell Labs President Gold Award, the Highest Honor in
%Recognition of R\&D Teams in Bell Labs, 2001
%
%\item Paul E. Torgersen Research Competition Finalist, Virginia Tech, 2009
%
%\item NSF Student Travel Grant to ACM MobiHoc 2009
%
%\item Student Travel Grant to IEEE MILCOM 2008
%
%\item NSF Student Travel Grant to IEEE INFOCOM 2008
%
%\item NSF Student Travel Grant to IEEE ISIT 2007
%\end{enumerate}
%
%%\vspace {.08in} \item Excellent Instructor of Bell Labs China Wireless Training, Bell Labs China, 2001.
%
%%\vspace {.08in} \item Midea Prize (8 out of 500 Graduate Students, Merit-based), South China Univ.
%%of Tech., 1999
%%
%%\vspace {.08in} \item First Class Scholarship for Three Consecutive Years, South China Univ. of
%%Tech., 1996-1999
%
%%\vspace {.08in} \item No.1 Overall out of Over 500 Examinees in Graduate School Admission Entrance Exam, South China Univ. of Tech., 1996
%
%%\vspace {.08in} \item Winner of the Contest of Electronics Design in SCUT, South China University
%%of Technology, 1995.
%
%%\vspace {.08in} \item Winner of the Contest of Advanced Mathematics, South China University
%%of Technology, 1993
%
%%\vspace {.08in} \item Excellent All-around Student, First Prize, South China Univ. of Tech. Three Consecutive Years, 1992-1994
%
%%\vspace{-.15in}
%
%%---------------------------------------------------------------------------
%%\vspace{.1in}
%\section{\bf \large Teaching Experience}
%\vspace{.1in}
%{\bf Guest Lecturer}, {The Ohio State University, March 2015} \\
%{\em Course:} Network Optimization and Algorithms \\
%\vspace{-.2in}
%
%{\bf Guest Lecturer}, {The Ohio State University, December 2014} \\
%{\em Course:} Introduction to Computer and Communication Networks \\
%\vspace{-.2in}
%
%{\bf Teaching Assistant}, {Virginia Tech, 2005}\\
%{\em Course:} Introduction to Telecommunication Networks \\
%{\em Responsibilities:} Giving lectures, grading assignments and exams
%
%
%
%%\vspace{-.1in}\\
%\section{\bf \large Current Research Topics}
%\hspace{1in} \begin{list2}
%\vspace{-.1in}
%
%\item {\bf Massive MIMO Wireless Network Control and Optimization:}
%    In this research, our goal is to bridge the gap between physical layer advances in Massive MIMO (an enabling technology for 5G wireless communications) and wireless networking research.
%    Specific research tasks include: i) Efficient scheduling design for Massive MIMO cellular networks, ii) Optimal routing and congestion control for Massive MIMO multi-hop backhaul networks, and iii) Energy analytics for Massive MIMO wireless networks.
%    Research results have appeared in IEEE Transactions on Mobile Computing (journal publication No.2), IEEE Journal on Selected Areas in Communications (journal publications No.6, No.10, No.11), IEEE INFOCOM (conference publications No.6 ({\bf Best Paper Runner-up}), No.7, No.15), ACM MobiHoc (conference publications No.10), IEEE ICC (conference publications No.13, {\bf Best Paper Award}), IEEE GLOBECOM (conference publication No.18), IEEE ISIT (conference publication No.16), IEEE MILCOM (conference publication No.12), IEEE WCNC (conference publication No.17).
%    This research has recently won a 3-year NSF grant from the CNS-NeTS program. The applicant is the {\bf sole PI} of this project.\\
%
%
%\item {\bf Distributed Control and Optimization for Wireless Networks Based on Second-Order Information}:
%    In this research, we focus on developing fast-converging and low-delay distributed algorithms for large-scale complex networks control and optimization that leverage second-order information (SOI).    
%    This research is built upon an SOI-based distributed network control and optimization theory recently developed by me that leverages SOI to significantly reduce delay and increase convergence speed.    
%    Results in this area have appeared in IEEE/ACM Transactions on Networking (journal publication No.1), ACM Sigmetrics Performance Evaluation Review (journal publication No.4), IEEE INFOCOM '13 and '12 (conference publications No.2 \& 3) and recently received a {\bf Best Paper Runner-up Award} from IEEE INFOCOM '13 (out of 1600+ submissions, acceptance rate 17\%).
%   Results in this research has also been supported in part by AFRL 2015 Visiting Faculty Research Program at the Information Directorate (AFRL/RI) and awarded the 2015 Air Force Summer Extension Grant.\\
%
%
%
%\item {\bf Indoor Cyber-Physical Network Infrastructure Optimization for Future Smart Buildings:} In this {\em interdiscipilneary} research, we consider jointly optimizing base station placement and power control to prolong the lifetime of wireless networks under harsh indoor wireless channels (due to floor/wall separations).
%    We propose both global optimization and novel efficient algorithms that targets at large-sized network infrastructure in buildings.
%    We focus on several performance objectives, including net-zero energy consumption, building safety and security, and in-door pervasive sensor network design and optimization.
%    Results in this research have appeared in IEEE Journal on Selected Areas in Communications (journal publication No.7), IJSNET (journal publication No.3), ASHRAE Transaction (journal publication No.8), IEEE INFOCOM (conference publications No.4), WASA (conference publication No.5)
%    Results in this research have also contributed to winning a 3-year NSF grant from the CPS program.\\
%
%
%\item {\bf Fast Disaster Response and Recovery for Future Smart Grid:}
%    In this research, we study interior-point based second-order distributed load shedding algorithms for disasters recovery in smart grid.
%    Contrary to existing gradient based algorithms that could constantly violate the constraints, our interior-point based approach guarantees feasibility at all times.
%    We propose a rooted spanning tree based reformulation, based on which we propose a double Sherman-Morrison-Woodbury (SMW) scheme that yields distributed computation schemes for primal Newton directions and dual variables.
%    We also design an efficient scheme to initialize our second-order load shedding algorithm and propose a simplified step-size selection strategy, which is well suited for implementations in practice and offers near-optimality guarantee.
%    Results in this research have appeared in ACM SIGMETRICS (conference publication No.1).
%    This research has also contributed in winning a 3-year NSF grant from the ECCS-CCSS program.
%
%
%\end{list2}
%
%%------------------------------------------------------------------------------
%\vspace{.1in}
%\section{\bf \large Publications}
%\vspace{.08in}
%\hspace{-.55in} {\bf Refereed Journal Articles}
%\vspace{.08in}
%
%\begin{enumerate}
%\item \textbf{Jia Liu}, Ness B. Shroff, Cathy H. Xia, and Hanif D. Sherali, ``Joint Congestion Control and Routing Optimization: An Efficient Second-Order Distributed Approach,'' accepted by {\em IEEE/ACM Transaction on Networking} in 2015, to appear.
%
%\vspace {.08in} \item Yi Shi, \textbf{Jia Liu}, Canming Jiang, Cunhao Gao, and Y. Thomas Hou, ``A DoF-Based Link Layer Model for Multi-Hop MIMO Networks,'' accepted for publication in {\em IEEE Transactions on Mobile Computing}, vol.~13, no.~7, pp.~1395-1408, Jul. 2014.
%
%\vspace {.08in} \item \textbf{Jia Liu}, Tianyou Kou, Qian Chen, and Hanif D. Sherali, ``On Wireless Network Infrastructure Optimization for Cyber-Physical Systems in Future Smart Buildings,'' {\em International Journal on Sensor Networks}, special issue on Internet of Things (IoT), vol.~18, no.~3-4, pp.~148-160, 2015.
%
%\vspace*{.08in} \item \textbf{Jia Liu}, Cathy H. Xia, Ness B. Shroff, and Xiaodong Zhang, ``On Distributed Computation Rate Optimization for Deploying Cloud Computing Programming Frameworks,'' {\em ACM Sigmetrics Performance Evaluation Review}, vol.~40, no.~4, pp.~63-72, Mar. 2013.
%
%\vspace {.08in} \item Yi Shi, Y. Thomas Hou, \textbf{Jia Liu}, and Sastry Kompella, ``Bridging the Gap between Protocol and Physical Models for Wireless Networks,'' {\em IEEE Transactions on Mobile Computing}, vol.~12, no.~7, pp.~1404-1416, Jul. 2013.
%
%\vspace {.08in} \item \textbf{Jia Liu}, Ness B. Shroff, and Hanif D. Sherali, ``Optimal Power Allocation in Multi-Relay MIMO Cooperative Networks: Theory and Algorithms,'' {\em IEEE Journal on Selected Areas in Communications}, vol.~30, no.~2, pp.~331-340, Feb. 2012.
%
%\vspace {.08in} \item \textbf{Jia Liu}, Tianyou Kou, Qian Chen, and Hanif D. Sherali, ``Femtocell Base Station Placement in Commercial Buildings: A Global Optimization Approach,'' {\em IEEE Journal on Selected Areas in Communications}, vol.~30, no.~3, pp.~652-663, Apr. 2012.
%
%\vspace {.08in} \item Sushant Sharma, Yi Shi, \textbf{Jia Liu}, Y. Thomas Hou, and Sastry Kompella, ``Network Coding in Cooperative Communications: Friend or Foe?'' {\em IEEE Transactions on Mobile Computing}, vol.~11, no.~7, pp.~1073-1085, Jul. 2012.
%
%\vspace {.08in} \item Hui Li, Lingying Zhao, Peter Ling, and \textbf{Jia Liu}, ``A Model for Predicting Wireless Signal Transmission Performance of ZigBee-Based Sensor Networks in Residential Houses,'' {\em ASHRAE Transactions}, vol. 118, no. 1, pp. 994-1007, Jan. 2012.
%
%
%\vspace*{.08in} \item \textbf{Jia Liu}, Y. Thomas Hou, Yi Shi, and Hanif D. Sherali, ``Cross-Layer Optimization on Routing and Power Control of MIMO Ad Hoc Networks: Routing, Power Allocation, and Bandwidth Allocation,'' {\em IEEE Journal on Selected Areas in Communications}, vol. 26, no. 6, pp. 913-926, Aug. 2008.
%
%\vspace {.08in} \item \textbf{Jia Liu}, Y. Thomas Hou, Yi Shi, and Hanif D. Sherali, ``On the Capacity of Multiuser MIMO Networks with Interference,'' {\em IEEE Transaction on Wireless Communications}, vol. 7, no. 2, pp. 488 - 494, Feb. 2008.
%
%\vspace {.08in} \item \textbf{Jia Liu}, Qian Chen, and Hanif D. Sherali, ``Algorithm Design for In-door Wireless Network Infrastructure,'' submitted to {\em IEEE Transaction on Mobile Computing}.
%
%\vspace {.08in} \item \textbf{Jia Liu}, Ness B. Shroff, and Hanif D. Sherali, ``Second-Order Distributed Cross-Layer Optimization, Part I: Wireline Networks,'' submitted to {\em IEEE/ACM Transaction on Networking}.
%
%\vspace {.08in} \item \textbf{Jia Liu}, Ness B. Shroff, and Hanif D. Sherali, ``Second-Order Distributed Cross-Layer Optimization, Part II: Wireless Networks,'' submitted to {\em IEEE/ACM Transaction on Networking}.
%
%\vspace {.08in} \item \textbf{Jia Liu}, Cathy H. Xia, Ness B. Shroff, ``Distributed Optimal Load Shedding for Disaster Recovery in Smart Electric Power Grids,'' submitted to {\em IEEE Transactions on Smart Grid}.
%
%%\vspace*{.08in} \item \textbf{Jia Liu}, Y. Thomas Hou and Hanif D. Sherali, ``Weighted Proportional Fairness Capacity of Gaussian MIMO Broadcast Channels,'' submitted to {\em IEEE Transaction on Information Theory}.
%%
%%\vspace*{.08in} \item \textbf{Jia Liu} and Y. Thomas Hou, ``On Routing and Power Allocation for Multi-hop MIMO Ad Hoc Networks with Dirty Paper Coding,'' submitted to {\em IEEE/ACM Transaction on Networking}.
%%
%%\vspace*{.08in} \item \textbf{Jia Liu} and Y. Thomas Hou, ``Maximum Weighted Sum Rate of Multi-Antenna Broadcast Channels,'' submitted to {\em IEEE Transaction on Wireless Communications}.
%%
%%\vspace*{.08in} \item \textbf{Jia Liu} and Y. Thomas Hou, ``Optimal Downlink Power Allocation and Scheduling for MIMO-Based WiMAX Access Networks,'' submitted to {\em IEEE Transaction on Wireless Communications}.
%
%%\vspace*{.08in} \item \textbf{Jia Liu} and Y. Thomas Hou, ``MIMO Ad Hoc Networks Under Imperfect Channel State Information,'' submitted to {\em IEEE Transaction on Wireless Communications}.
%
%\end{enumerate}
%
%%---Conference Papers---
%\hspace{-.55in} {\bf Refereed Conference Papers}
%\vspace*{.08in}
%\begin{enumerate}
%
%\vspace*{.08in} \item \textbf{Jia Liu}, Atilla Eryilmaz, Ness B. Shroff, and Elizabeth S. Bentley, ``Heavy-Ball: A New Approach to Tame Delay and Convergence in Wireless Network Optimization,'' in {\em Proc. IEEE INFOCOM} 2016, accepted, to appear.
%
%\vspace*{.08in} \item \textbf{Jia Liu}, Cathy H. Xia, Ness B. Shroff, and Hanif D. Sherali, ``Distributed Optimal Load Shedding for Disaster Recovery in Smart Electric Power Grids: A Second-Order Approach,'' to appear in {\em Proc. ACM Sigmetrics 2014}.
%
%\vspace*{.08in} \item \textbf{Jia Liu}, Cathy H. Xia, Ness B. Shroff, and Hanif D. Sherali, ``Distributed Cross-Layer Optimization in Wireless Networks: A Second-Order Approach,'' in {\em Proc. IEEE INFOCOM 2013}, Turin, Italy, Apr. 14-19, 2013 ({\bf Best Paper Runner-up Award}).
%
%%\vspace*{.08in} \item \textbf{Jia Liu}, Cathy H. Xia, and Ness B. Shroff, ``A Second-Order Approach for Distributed In-Network Computing in Computation Analytics,'' submitted to ACM Sigmetrics 2013.
%
%
%\vspace*{.08in} \item \textbf{Jia Liu} and Hanif D. Sherali, ``A Distributed Newton's Method for Joint Multi-Hop Routing and Flow Control: Theory and Algorithm,'' in {\em Proc. IEEE INFOCOM}, Orlando, FL, Mar. 25 - 30, 2012.
%
%\vspace*{.08in} \item \textbf{Jia Liu}, Qian Chen, and Hanif D. Sherali, ``Algorithm Design for Femtocell Base Station Placement in Commercial Building Environments,'' in {\em Proc. IEEE INFOCOM}, Orlando, FL, Mar. 25 - 30, 2012.
%
%\vspace*{.08in} \item \textbf{Jia Liu}, Tianyou Kou, Qian Chen, and Hanif D. Sherali, ``On Wireless Network Infrastructure Optimization for Cyber-Physical Systems in Future Smart Buildings,'' in {\em Proc. IEEE WASA}, Yellow Mountains, China, Aug. 8-10, 2012.
%
%
%\vspace*{.08in} \item Yi Shi, \textbf{Jia Liu}, Canming Jiang, Cunhao Gao, and Y. Thomas Hou, ``An Optimal Link Layer Model for Multi-hop MIMO Networks,'' in {\em Proc. IEEE INFOCOM 2011} ({\bf Best Paper Runner-up Award}), Shanghai, China, Apr. 10 - 15, 2011.
%
%\vspace*{.08in} \item \textbf{Jia Liu}, Yi Shi, and Y. Thomas Hou, ``A Tractable and Accurate Cross-Layer Model for Multi-Hop MIMO Ad Hoc Networks,'' in {\em Proc. IEEE INFOCOM 2010}, San Diego, CA, Mar. 15 - 19, 2010.
%
%\vspace*{.08in} \item S. Sushant, Yi Shi, \textbf{Jia Liu}, Y. Thomas Hou, and Sastry Kompella``Is Network Coding Always Good for Cooperative Communications?'' in {\em Proc. IEEE INFOCOM 2010}, San Diego, CA, Mar. 15 - 19, 2010.
%
%\vspace*{.08in} \item \textbf{Jia Liu}, Y. Thomas Hou, Yi Shi, and Hanif D. Sherali, ``On Performance Optimization for Multi-Carrier MIMO Ad Hoc Networks,'' in {\em Proc. ACM MobiHoc 2009}, New Orleans, LA, May 18 - 21, 2009.
%
%\vspace*{.08in} \item Yi Shi, Y. Thomas Hou, \textbf{Jia Liu}, and Hanif D. Sherali, ``How to Correctly Use the Protocol Interference Model for Multi-hop Wireless Networks,'' in {\em Proc. ACM MobiHoc 2009}, New Orleans, LA, May 18 - 21, 2009.
%
%\vspace*{.08in} \item \textbf{Jia Liu}, Y. Thomas Hou, and Hanif D. Sherali, ``Optimal Power Allocation for Achieving Perfect Secrecy Capacity in MIMO Wire-Tap Channels,'' in {\em Proc. 43rd Annual Conference on Information Sciences and Systems (CISS) 2009}, Baltimore, MD, Mar. 18 - 20, 2009.
%
%\vspace*{.08in} \item \textbf{Jia Liu}, Y. Thomas Hou, and Hanif D. Sherali, ``On the Performance of MIMO-Based Ad Hoc Networks under Imperfect CSI,'' in {\em Proc. IEEE MILCOM 2008}, San Diego, CA, Nov. 17 - 19, 2008.
%
%\vspace*{.08in} \item \textbf{Jia Liu}, Y. Thomas Hou, and Hanif D. Sherali, ``Cross-Layer Optimization for MIMO-Based Mesh Networks with Dirty Paper Coding,'' in {\em Proc. IEEE ICC 2008} ({\bf Best Paper Award}), Beijing, China, May 19 - 23, 2008.
%
%\vspace*{.08in} \item \textbf{Jia Liu}, Y. Thomas Hou and Hanif D. Sherali, ``Maximum Weighted Sum Rate of Multi-Antenna Broadcast Channels,'' in {\em Proc. IEEE ICC 2008}, Beijing, China, May 19 - 23, 2008.
%
%\vspace*{.08in} \item \textbf{Jia Liu} and Y. Thomas Hou, ``Weighted Proportional Fairness Capacity of Gaussian MIMO Broadcast Channels,'' in {\em Proc. IEEE INFOCOM 2008}, Phoenix, AZ, Apr. 13 - 17, 2008.
%
%\vspace*{.08in} \item \textbf{Jia Liu}, Y. Thomas Hou, and Hanif D. Sherali, ``Conjugate Gradient
%Projection Approach for MIMO Gaussian Broadcast Channels,'' in {\em Proc. IEEE ISIT}, Nice, France,
%Jun. 24 - 29, 2007.
%
%\vspace*{.08in} \item \textbf{Jia Liu}, T. Y. Park, Y. Thomas Hou, Yi Shi, and Hanif D. Sherali,
%``Cross-Layer Optimization of MIMO-Based Mesh Networks Under Orthogonal Channels,'' in {\em Proc.
%IEEE WCNC}, Hong Kong, Mar. 11 - 15, 2007.
%
%\vspace*{.08in} \item \textbf{Jia Liu}, Y. Thomas Hou, Yi Shi, and Hanif D. Sherali, ``Optimization
%of Multiuser MIMO Networks with Interference,'' {\em Proc. IEEE GLOBECOM}, San Franscisco, Nov. 27
%- Dec. 1, 2006.
%
%\vspace*{.08in} \item \textbf{Jia Liu} and A. Annamalai, ``Efficacy of Channel-and-Node Aware
%Routing Strategies in Wireless Ad Hoc Networks,'' in {\em Proc. IEEE VTC}, Dallas, Oct. 2005.
%
%\vspace*{.08in} \item \textbf{Jia Liu} and A. Annamalai, ``Channel-Aware routing Protocol for Ad
%Hoc Networks: Generalized Multiple-Route Path Selection Diversity,'' in {\em Proc. IEEE VTC},
%Dallas, Oct. 2005.
%
%\vspace*{.08in} \item A. Annamalai and \textbf{Jia Liu}, ``A Cross-Layer Design Perspective for
%Multi-Resolution Signaling,'' in {\em Proc. IEEE GLOBECOM}, Dallas, Nov. 2004.
%
%\vspace*{.08in} \item \textbf{Jia Liu} and A. Annamalai, ``Multi-Resolution Signaling for
%Multimedia Multicasting,'' in {\em Proc. IEEE VTC}, Los Angeles, Sep. 2004.
%
%\end{enumerate}
%
%
%
%\section{\bf \large Research Grants}
%\vspace{.16in}
%\begin{enumerate}
%
%\vspace*{.08in} \item ``NeTS: Small: Toward Optimal, Efficient, and Holistic Networking Design for Massive-MIMO Wireless Networks,'' NSF CNS-1527078, \$300,000, 10/01/2015 -- 09/30/2018 (sole PI).
%
%\vspace*{.08in} \item ``Dynamic Resource Allocation for Airborne Networks under Spectral, Spatial, and Temporal Uncertainty,'' AFRL 2015 Visiting Faculty Research Program Award, \$15,000, 06/22/2015 -- 08/28/2015 (sole PI).
%
%\vspace*{.08in} \item ``Dynamic Resource Allocation for Airborne Networks under Spectral, Spatial, and Temporal Uncertainty,'' AFRL 2015 Summer Extension Grant, \$10,000, 08/29/2015 -- 10/31/2015 (sole PI).
%
%\vspace*{.08in} \item ``CPS: Synergy: Collaborative Research: Cognitive Green Building: A Holistic Cyber-Physical Analytic Paradigm for Energy Sustainability,'' NSF CNS-1446582, \$1,000,000, 01/01/2015 -- 12/31/2017 (internal Co-PI, PI: Ness B. Shroff, Co-PIs: Qian Chen, Thomas Hou, Wenjing Lou).
%
%%\vspace*{.08in} \item ``CyberSEES: Type 2: Cyber-Enabled Sustainable Management for the Electric Power Grid under Extreme Events,'' pending (senior personnel and main technical contributor, PI: Cathy H. Xia, Co-PI: Ness B. Shroff and Ramteen Sioshansi).
%
%%\vspace*{.08in} \item ``NeTS: Small: Dynamic Spectrum Access under Uncertainty: Theory, Algorithm Development, and Evaluation,''  NSF CNS-1421576 (senior personnel and main technical contributor, PI: Ness B. Shroff, Co-PI: Eylem Ekici).
%
%%\vspace*{.08in} \item ``NeTS: Medium: Energy Efficient Operation and Control of Green Base Stations with Renewable Energy: Theory to Practice,'' NSF CNS-1409336 (senior personnel and main technical contributor, PI: Prasun Sinha, Co-PI: Ness B. Shroff).
%
%\vspace*{.08in} \item ``ECCS: Toward Efficient and Distributed Cyber-Physical Systems Design for the Smart Electric Power Grid,'' NSF ECCS 1232118, \$431,067, 09/01/2012--08/31/2016 (senior personnel and main technical contributor, PI: Cathy H. Xia, Co-PI: Ness B. Shroff).
%
%\vspace*{.08in} \item ``Exploring Performance Limits of Multi-hop MIMO Networks via Tractable Models and Optimization,'' NSF ECCS 1102013, \$400,000, Jul. 2011 (assistant and technical contributor, PI: Thomas Hou).
%
%\vspace*{.08in} \item ``NeTS-WN: Capacity Problems for MIMO-Enabled Wireless Mesh Networks,'' NSF CNS 0721421, \$359,158, Sep. 2007 (assistant and technical contributor, PI: Thomas Hou).
%
%\vspace*{.08in} \item ``Novel Multi-hop Networking Technologies for MIMO Systems in Tactical Communication Networks,'' ONR N00014-08-0084, \$290,000, Dec. 2007 (assistant and technical contributor, PI: Thomas Hou).
%
%%\vspace*{.08in} \item ``A CPS Approach to Future Net-Zero Energy, Cognitive, and Adaptive Building,'' (tentative title), to be submitted to NSF CISE CPS Program, estimated \$2,000,000, Jan. 2012 (Co-PI, with Prof. Thomas Hou, Prof. Wenjing Lou, Prof. Scott Midkiff, and Prof. Qian Chen).
%%%({\em Note: Tom, I have been granted PI eligibility at OSU as a PostDoc. Could I be listed as a Co-PI in this proposal? (I won't claim any share of budget in this proposal)})
%%
%%\vspace*{.08in} \item ``Distributed Second-Order Approaches for Cross-Layer Optimization in Wireless Networks,'' (tentative title), to be submitted to NSF ECCS, estimated \$1,000,000, Jan. 2012 (Co-PI, with Prof. Ness Shroff and Prof. Honghui Xia).
%
%\end{enumerate}
%
%
%
%%\section{\bf \large United States Patent}
%%\vspace{.16in}
%%\begin{enumerate}
%%\vspace*{.08in} \item ``An Efficient Method for Interference Cancellation and Spatial Multiplexing in MIMO Wireless Networks,'' US Patent Application No. 61/309,902, March, 2010.
%%\end{enumerate}
%
%%------------------------------------------------------------------------------
%\section{\bf \large Invited Talks and Conference Presentations}
%\vspace {.1in} Jia Liu (2013). ``Toward Optimal, Efficient, and Distributed System Design for the Smart Electric Power Grid,'' May. 2013, OSU-Battelle Smart Grid Collaborative Workshop.
%
%\vspace {.1in} Jia Liu (2013). ``Heterogeneous Delay Tolerant Task Scheduling and Energy Management in the Smart Grid with Renewable Energy,'' May. 2013, OSU-Battelle Smart Grid Collaborative Workshop.
%
%\vspace {.1in} Jia Liu (2013). ``Distributed Cross-Layer Optimization: A Second-Order Approach,'' IEEE INFOCOM, Apr. 2013, Turin, Italy.
%
%\vspace {.1in} Jia Liu (2012). ``Second-Order Distributed Algorithms for
%Networked Systems Optimization: Theory and Applications,'' IBM T. J. Watson Research Center, Oct. 23, Yorktown Heights, NY.
%
%\vspace {.1in} Jia Liu (2012). ``A Distributed Newton's Method for Joint Multi-Hop Routing and Flow Control: Theory and Algorithm,'' IEEE INFOCOM, Mar. 2012, Orlando, FL.
%
%\vspace {.1in} Jia Liu (2012). ``On Wireless Network Infrastructure Optimization for Cyber-Physical Systems in Future Smart Buildings,'' IEEE INFOCOM, Mar. 2012, Orlando, FL.
%
%\vspace {.1in} Jia Liu (2010). ``A Tractable and Accurate Cross-Layer Model for Multi-Hop MIMO Ad Hoc Networks,'' IEEE INFOCOM, Mar. 2010, San Diego, CA.
%
%\vspace {.1in} Jia Liu (2009). ``Toward A Theoretical Foundation for MIMO-Based Networks Optimization,'' the 1st ACM MobiHoc S$^3$ Workshop, May. 17, New Orleans, LA (invited).
%
%\vspace {.00in} Jia Liu (2009). ``On Performance Optimization for MC-MIMO Ad Hoc Networks,'' the 10th ACM International Symposium on Mobile Ad Hoc Networking and Computing (MobiHoc), May. 19, New Orleans, LA.
%
%\vspace {.00in} Jia Liu (2009). ``Optimal Power Allocation for Achieving Perfect Secrecy Capacity in MIMO Wire-Tap Channels,'' 43rd Annual Conference on Information Sciences and Systems (CISS), Mar. 19, Baltimore, MD.
%
%\vspace {.00in} Jia Liu (2008). ``On the Performance of MIMO-Based Ad Hoc Networks under Imperfect CSI,'' IEEE MILCOM, Nov. 18, San Diego, CA.
%
%\vspace {.00in} Jia Liu (2008). ``Cross-Layer Optimization for MIMO-Based
%Mesh Networks with Dirty Paper Coding,'' IEEE ICC, May 21, Beijing, China.
%
%\vspace {.00in} Jia Liu (2008). ``Maximum Weighted Sum Rate of Multi-Antenna Broadcast Channels,'' IEEE ICC, May 22, Beijing, China.
%
%\vspace {.00in} Jia Liu (2008). ``Weighted Proportional Fairness Capacity of Gaussian MIMO Broadcast Channels,'' IEEE INFOCOM, April 16, Phoenix, AZ.
%
%\vspace {.00in} Jia Liu (2007). ``Conjugate Gradient Projection Approach for MIMO Gaussian Broadcast Channels,'' IEEE ISIT, June 26, Nice, France.
%
%\vspace {.00in} Jia Liu (2006). ``Optimization of Multiuser MIMO Networks with Interference,'' IEEE
%Globecom, December 1, San Francisco, CA.
%
%\vspace {.00in} Jia Liu (2004). ``Multi-Resolution Signaling for Multimedia Multicasting,'' IEEE
%VTC, September 28, Los Angeles, CA.
%
%
%%%%---------------------------------------------------------------------------%%
%\section{\bf \large Mathematical Skills}
%\vspace{.25in}
%\begin{itemize}
%\item Optimization (Linear Programming, Nonlinear Programming, Integer Programming, Dynamic Programming)
%\item Graph Theory and Network Flows
%\item Probability and Statistics, Stochastic Processes
%\item Linear and Nonlinear Control Theory
%\item Matrix Analysis and Theory
%\end{itemize}
%
%
%\section{\bf \large Technical Skills}
%\begin{description}
%\item[Computer Languages:] C/C++/C\#, JAVA, Perl, PHP, Unix Shell, HTML, \LaTeX
%\item[Software Tools:] MATLAB, Mathematica, Visual C++, LINDO API, CPLEX
%\end{description}
%%------------------------------------------------------------------------------
%
%
%\section{\bf \large Affliations/Memberships}
%\vspace{.1in} IEEE Member/IEEE ComSoc Member (since 2003)
%
%\vspace{-.0in} Member of Society for Industrial and Applied Mathematics (SIAM) (since 2007)
%
%\vspace{-.0in} Member of Tau Beta Pi, the National Engineering Honor Society (since 2006)
%
%\vspace{-.0in} Member of Eta Kappa Nu (HKN), Electrical and Computer Honor Society (since 2008)
%
%\section{\bf \large Professional Services}
%\vspace{.16in}
%\hspace{-.55in} {\bf Reviewer for Journals}
%\vspace{.16in}
%\begin{itemize}
%\item IEEE Transactions on Information Theory (2009)
%\item IEEE Transactions on Mobile Computing (2009, 2010)
%\item IEEE Transactions on Wireless Communications (2004, 2005, 2006, 2007, 2008, 2009, 2010)
%\item IEEE Transactions on Vehicular Technologies (2004, 2005, 2006, 2007, 2012)
%\item IEEE Journal on Selected Areas in Communications (2010, 2011, 2012)
%\item EURASIP Journal on Advances in Signal Processing (2008)
%\end{itemize}
%
%\hspace{-.55in} {\bf Reviewer for Conferences}
%\vspace{.16in}
%\begin{itemize}
%\item IEEE INFOCOM (2007--2016)
%\item IEEE ICC (2007, 2008, 2009, 2010)
%\item IEEE GLOBECOM (2004, 2009)
%\item IEEE WCNC (2005, 2007)
%\item IEEE SECON (2005)
%\item VTC (2004, 2005)
%\end{itemize}
%
%
%\hspace{-.55in} {\bf Technical Program Committee (TPC) Member}
%\vspace{.16in}
%\begin{itemize}
%\item IEEE INFOCOM (2010--2015)
%\item IEEE ICCCN (2013)
%\item IEEE WCNC (2013, 2014)
%\item IEEE WASA (2011--2012)
%\item IEEE CNS (2013)
%\end{itemize}
%
%%------------------------------------------------------------------------------
%
%%\section{\bf \large Related Courses}
%%\vspace{.1in}
%%\begin{tabular}{@{}p{.75in}p{3.5in}p{2in}}
%%BC 5984 & Principles Construction I & A \\
%%BC 5984 & Principles Construction II & A- \\
%%BC 5984 & Construction Practice I & A \\
%%BC 4444 & Construction Practice II & A \\
%%BC 5244 & Preservation: Practices & A \\
%%BC 5264 & Fundamentals Indoor Air Science & A \\
%%BC 5984 & Sustainable Building Construction & A \\
%%ARCH 5045 & Environmental Design Research & A \\
%%CE 5602 & Project Management \& Site Control & A \\
%%CE 5603 & Engineering Economy \& Project Evaluation & B+ \\
%%BC 6064 & Past, Present, Future AEC & A \\
%%EDP 6005 & Seminar Environmental Planning - Research & A \\
%%EDP 6006 & Seminar Environmental Planning - Teaching & A- \\
%%\end{tabular}
%
%\newpage
%\medskip
\section{\bf \large References}
\vspace{.1in}
{\bf Dr. Ness B. Shroff} \\
Chaired Professor of ECE and CSE, IEEE Fellow \\
Ohio Eminent Scholar in Networking and Communications \\
Department of Electrical and Computer Engineering \\
The Ohio State University \\
764 Dreese Laboratory, Columbus, OH 43210\\
Tel: (614) 247-6554 \\
Email: shroff.11@osu.edu\\
({\em Postdoc Advisor}) 


\medskip
{\bf Dr. Y. Thomas Hou}                  \\
Bradley Distinguished Professor of ECE, IEEE Fellow \\
Department of Electrical and Computer Engineering \\
Virginia Polytechnic Institute and State University    \\
2040-M Torgersen Hall, Blacksburg, VA 24061  \\
Tel: (540) 231-2950 \\
Email: thou@vt.edu \\
({\em Ph.D. Advisor})


\medskip
{\bf Dr. Atilla Eryilmaz} \\ %(Ph.D. Committee Member) \\
Associate Professor  \\
Department of Electrical and Computer Engineering \\
The Ohio State University \\
760 Dreese Laboratory, Columbus, OH 43210 \\
Tel: (614) 292-7464 \\
Email: eryilmaz.2@osu.edu



\medskip
{\bf Dr. Jeffrey H. Reed} \\ %(Ph.D. Committee Member) \\
W. Worcester Endowed Professor, IEEE Fellow  \\
Department of Electrical and Computer Engineering \\
Virginia Polytechnic Institute and State University \\
439 Durham Hall, Blacksburg, VA 24061 \\
Tel: (540) 231-2972 \\
Email: reedjh@vt.edu


\medskip
{\bf Dr. John L. Volakis} \\ %(Ph.D. Committee Member) \\
Chope Chair Professor, IEEE Fellow  \\
Department of Electrical and Computer Engineering \\
The Ohio State University \\
1330 Kinnear Road and Dreese Labs, Room 360, Columbus, OH 43210 \\
Tel: (614) 292-5846 \\
Email: volakis.1@osu.edu



%\medskip
%{\bf Dr. Cathy H. Xia} \\ % (Ph.D. Committee Member) \\
%Associate Professor                          \\
%Department of Integrated Systems Engineering \& \\
%(by courtesy) Department of Computer Science and Engineering \\
%The Ohio State University \\
%242 Baker Systems Engineering, Columbus, OH 43210 \\
%Tel: (614) 249-6985 \\
%Email: xia.52@osu.edu

%\medskip
%{\bf Dr. Jeffrey H. Reed} \\ %(Ph.D. Committee Member) \\
%W. Worcester Endowed Professor, IEEE Fellow  \\
%Department of Electrical and Computer Engineering \\
%Virginia Polytechnic Institute and State University \\
%439 Durham Hall, Blacksburg, VA 24061 \\
%Tel: (540) 231-2972 \\
%Email: reedjh@vt.edu






\end{resume}

\end{document}

%%===========================================================================%%
